\section{Consequences of Bad Planning}
Bad planning in a business aspect has an array of consequences. The direct effects of bad planning, in the context of ground handling companies, could be inefficient schedules for the ground handlers, not being properly prepared for emergencies and/or difficulties attaining performance reports on the ground handlers.

These direct consequences do not stand for themselves. Each of these cases give rise to an entirely new sprout of indirect consequences.
Looking at the case of inefficient schedules for the ground handlers, we need to define what it means for a schedule to be inefficient. Here there are three different cases
\begin{itemize}
	\item The duties on the schedules could be too close to each other
\end{itemize}
This means that the ground handler does not have the time to finish the first task without delaying the next. Consequences for this case includes amongst others, stress for the ground handler, delays in the overall handling schedule for the day and less productive days than anticipated by the supervisors.
\begin{itemize}
	\item The time it takes for a ground handler to move from task A to task B could be underestimated
\end{itemize}
This case has some of the same consequences as the first one. If the ground handlers suddenly has less time to do what he needs to than he had anticipated, he will become stressed and the second task would be delayed amongst.
\begin{itemize}
	\item The ground handler could be assigned to a task which he is inexperienced with
\end{itemize}
This could be a very dangerous situation, as many ground handlers do important jobs when it comes to aircraft operation and maintenance. If a vital task, such as making sure the landing gear works properly, is done badly, it puts the aircraft and it's passengers and crew at risk.


Another potential effect of bad planning would be not being properly prepared for an emergency. If the ground handlers' and emergency crews' schedules have been poorly designed, there is an enhanced risk that there will not be an ideal, or at least sufficient, number of crew available to perform the emergency protocols. If an emergency is handled incorrectly, the whole operation of the airport could be at risk.


A different aspect on poor planning, is the perspective of the supervisors. The supervisors need to understand how their crewmen perform, who is better at one task and who is better another. Doing so manually takes a lot of energy for the supervisors, who need to keep up with all of the individual ground handlers. If the ground handlers' schedules are poorly planned, the supervisor's view of the ground handlers would be affected by this. If the supervisor sees that one employee is always performing poorly, he would think that the employee is therefore a poor performer. In reality it is just as likely that the employee is not assigned his preferred task, or that he is stressed.


To avoid all of these consequences, a good software solution would not only make planning easier, it would make sure that the planning is done optimally.