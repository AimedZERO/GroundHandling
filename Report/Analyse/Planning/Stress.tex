\section{Stress}
When dealing with work related stress the biggest causes are normally organizational factors, and the solution is normally to gain control through prevention and management.

When trying to understand stress it is important to understand that a situation can only be potential stressful depending on the person experiencing it. That means that a situation can evoke stress for one person, but the same situation can have no effect for another, depending on their way of handling understanding and viewing the situation. Aside from the person normally timing frequency, intensity and duration, are factors that determinate how big of an impact a potential stressful situation can have on the person.

"Stress can also have a considerable organisational and economic impact."

HSE's  longitudinal studies of occupational stress shows that 

"HSE's  logitudinal studies of occupational stress using changes in naturally occurring work situations have provided evidence that over quite short periods of time the nature of the work situation to which the person is exposed has significant effects on mental health quite apart from contributions from personality and pre-existing psychological health."

When dealing with work related stress an organization first need to manage and treat people on an individual level meaning that they first need to treat the people that are already stressed.

After treating the individuals the organization need to approach its employees with interventions that restructuring the work flow. Organization of work tasks and inform their employees about the situation and how to deal with these changes. This approach is also very preventive for future problems, and can lead to, new work tasks and flow if it is done correctly, the potential stressful situations can be completely removed thereby removing the problem.

At this time the importance of training the individuals in managing potential stressful situation themselves, must be noted and it is largely accepted that this is the key for a healthy workforce. Also because interventions can be very costly in short term and may not even work or even make the problem worse, if for instance the new work plan is even more stressful.

Individual stress management focus on biofeedback, muscle relaxation and cognitive restructuring of appraisal and coping responses. This is a good approach because:

\begin{itemize}
\item They can quickly be evaluated and established without needing to change any organizational structure or work flow.
\item "They can encompass the need to take into account perceptions and reactions and are thus particularly appropriate for individual needs."
\item "They may be helpful in combating non-work as well as work-based stress problems (which may interact synergistically)."
\item "They can be incorporated into existing employee assistancehealth education packages."
\end{itemize}

One of the problems when dealing with stress as an individual, is to understand the complex situation with all the work and non-work related factors the contribute to ones mental health and their feelings of uneasiness or distress. Most people will need help from specialist to unravel this.

When combating stress in work-related situations on an individual, the organization can either help the person cope and handle poorly designed work methods or tasks or "the person to regain a degree of normal functioning so that return to productive working can be achieved."

"Moreover the impact of stress inoculation training may decrease over time leading to the need for repeated refresher training."

"The suggestion from some quarters is that stress management training should only be used to supplement organisational changd job redesign programmes inorder to deal with stressors which cannot be designed out of the job very easily, for example, seasonal workloads. Thus management at the secondary level should be used to supplement attempts to assess and restructure sources of stress in the work environment- organisational, ergonomic and psychosocial."


SOURCE: http://ieeexplore.ieee.org/Xplore/cookiedetectresponse.jsp