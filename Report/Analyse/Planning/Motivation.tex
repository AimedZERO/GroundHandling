\subsection{Motivation}
Activities that are not interesting (i.e., that are not intrinsically motivating) require extrinsic motivation, so their initial enactment depends upon the perception of a contingency between the behavior and a desired consequence such as implicit approval or tangible rewards.

When externally regulated, people act with the intention of obtaining a desired consequence or avoiding an undesired one, so they are energized into action only when the action is instrumental to those ends (e.g., I work when the boss is watching).

amotivation involves having no intentions for the behavior and not really knowing why one is doing it. Research using this assessment strategy has confirmed that, in domains such as education (Williams \& Deci, 1996),

We argued earlier that the needs for competence and autonomy underlie intrinsic motivation—that people need to feel competent and autonomous to maintain their intrinsic motivation—and experiments were reviewed that provided support for that proposition.

The need for relatedness—is also crucial for internalization (e.g., Baumeister \& Leary, 1995). More specifically, SDT postulates that when people experience satisfaction of the needs for relatedness and competence with respect to a behavior, they will tend to internalize its value and regulation, but the degree of satisfaction of the need for autonomy is what distinguishes whether identification or integration, rather than just introjection, will occur.

Using this definition, the needs for competence, autonomy, and relatedness are considered important for all individuals,

that work climates that promote satisfaction of the three basic psychological needs will enhance employees’ intrinsic motivation and promote full internalization of extrinsic motivation and that this will in turn yield the important work outcomes of (1) persistence and maintained behavior change; (2) effective performance, particularly on tasks requiring creativity, cognitive flexibility, and conceptual understanding; (3) job satisfaction; (4) positive work-related attitudes; (5) organizational citizenship behaviors; and (6) psychological adjustment and well-being.


Hackman and Oldham (1980) argued that the most effective means of motivating individuals is through the optimal design of jobs.

The authors proposed that the means for increasing internal work motivation is to design jobs so they will (1) provide variety, involve completion of a whole, and have a positive impact on the lives of others; (2) afford considerable freedom and discretion to the employee (what action theorists refer to as decision latitude); and (3) provide meaningful perfor-
mance feedback.

constructive feedback as one way to influence autonomous motivation, but it also suggests that the interpersonal style of supervisors and managers is important.

Pertinent to this is the finding that jobs with high motivating potential scores were associated with enhanced psychological states and better outcomes only for workers who perceived that pay and promotion were not contingent on performance (Johns, Xie, & Fang, 1992).

studies have found that managers’ autonomy support led to greater satisfaction of the needs for competence, relatedness, and autonomy and, in turn, to more job satisfaction, higher performance evaluations, greater persistence, greater acceptance of organizational change, and better psychological adjustment (Baard et al., 2004; Deci et al., 2001; Gagne´ et al., 2000; Ilardi, Leone, Kasser, \& Ryan, 1993; Kasser, Davey, & Ryan, 1992).

Bono and Judge (2003) showed that followers of transformational or visionary leaders were more likely to adopt autonomous goals than controlled goals in the workplace. These followers were also more satisfied with their jobs and more affectively committed to the organization.

Taken together, studies in organizations have provided support for the propositions that autonomy-
supportive (rather than controlling) work environments and managerial methods promote basic need satisfaction, intrinsic motivation, and full internalization of extrinsic motivation, and that these in turn lead to persistence, effective performance, job satisfaction, positive work attitudes, organizational commitment, and psychological well-being.


Together, the studies suggest that autonomous motivation, consisting of a mix of intrinsic motivation and internalized extrinsic motivation, is superior in situations that include both complex tasks that are interesting and less complex tasks that require discipline. When a job involves only mundane tasks, however, there appears to be no performance advantage to autonomous motivation. Still, even in those situations, autonomous motivation will be associated with greater job satisfaction and well-being, as was found by Ilardi et al. (1993) in a study of employees with monotonous jobs in a shoe factory and by Shirom and colleagues (1999) in a study of blue-collar workers with mundane jobs. This implies that, overall, autonomous motivation is preferable in organizations because even with dull, boring jobs there is an advantage to autonomous motivation in terms of job satisfaction and well-being, which are likely to yield better attendance and lower turnover (Breaugh, 1985; Karasek & Theorell, 1990; Matteson & Ivancevich, 1987; Sherman, 1989).

research suggests that intrinsic motivation (based in interest) and autonomous extrinsic motivation (based in importance) are both related to performance, satisfaction, trust, and well-being in the workplace.