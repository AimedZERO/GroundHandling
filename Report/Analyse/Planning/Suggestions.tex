\section{Suggestions}
    A couple of suggestions on how to boost safety and maintain on-time schedule, was concluded by (Aircraft-Ground-Handling-and-Human-Factors-NLR-final-report), the report is based on a investigation and questionnaires on both the management segment and the operational segment. According to operational personel the most frequent factor in directly causing time inflicting incidents, are the equipment in use at turn-around. As far personal factors to incidents, both components have unified opinion in that stress, fatigue and time pressure. These relate to most occuring delays and disruptions in operations of takeoffs and landings. To cope with time pressure, stress and fatigue, it is recommended to provide complementary training and evaluate if focus should be on on-time-departure or on-time-arrival to ease time-pressure on employees. Also suggested are standardizing terminology ensures that miscommunication occurs less often and finding a common opinion between management- and operation-organisation on maintenance of equipment and tools to deal with poor equipment factor. Introducing some these actions could result possibly reduce the number of incident during the ground handling process.

