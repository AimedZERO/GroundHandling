\section{Suggestions}
    A couple of suggestions on how to boost safety and maintain on-time schedules, were concluded by \cite{Bossenbroek2010}, the report is based on an investigation, and questionnaires, on both the management segment and the operational segment of an airport. According to operational personnel the most frequent factor in directly causing time consuming incidents, is equipment in use at turnaround. 
    Personal factors relation to incidents, both operational personnel and management have unified opinions in that stress, fatigue and time pressure is biggest contributors. These relate to most occurring delays and disruptions in operations of takeoffs and landings. To cope with time pressure, stress and fatigue, it is recommended to provide complementary training and evaluate if the focus should be on on-time-departure or on-time-arrival to ease time pressure on employees. Another suggestion is standardizing terminology ensuring that miscommunication occurs less often and finding a common opinion between management- and operation-organisation on maintenance of equipment and tools to deal with poor equipment factor. Introducing some of these actions could possibly result in reducing the number of incidents during the ground handling process.

\section{Conclusion on motivation and stress}
    Previous sections on stress, motivation and suggestion to solving some these, concludes that its possible to introduce preventive measures to ease and promote an effective work environment, but that stress and motivation is depended on the individual working. Some of these measure can be taken into consideration in formulating a product that accounts for these factors.