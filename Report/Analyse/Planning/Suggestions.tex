\section{Suggestions}
    A couple of suggestions on how to boost safety and maintain on-time schedules, were concluded by (Aircraft-Ground-Handling-and-Human-Factors-NLR-final-report), the report is based on an investigation, and questionnaires, on both the management segment and the operational segment. According to operational personnel the most frequent factor in directly causing time consuming incidents, is equipment in use at turn-around. Personal factors' relation to incidents, both components have unified opinion in that stress, fatigue and time pressure. These relate to most occurring delays and disruptions in operations of takeoffs and landings. To cope with time pressure, stress and fatigue, it is recommended to provide complementary training and evaluate if the focus should be on on-time-departure or on-time-arrival to ease time pressure on employees. Also suggested are standardizing terminology ensures that miscommunication occurs less often and finding a common opinion between management- and operation-organization on maintenance of equipment and tools to deal with poor equipment factor. Introducing some of these actions could possibly result in reducing the number of incident during the ground handling process.

