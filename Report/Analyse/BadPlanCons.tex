\section{Consequences of Bad Planning}
Bad planning in a business aspect has an array of consequences. The direct effects of bad planning, in the context of ground handling companies, could be inefficient schedules for the ground handlers, not being properly prepared for emergencies and/or difficulties attaining performance reports on the ground handlers.

These direct consequences do not stand for themselves. Each of these cases give rise to an entirely new sprout of indirect consequences.
Looking at the case of inefficient schedules for the ground handlers, we need to define what it means for a schedule to be inefficient. Here there are three different cases
\begin{itemize}
	\item The duties on the schedules could be too close to each other
\end{itemize}
This means that the ground handler does not have the time to finish the first task without delaying the next. Consequences for this case includes amongst others, stress for the ground handler, delays in the overall handling schedule for the day and less productive days than anticipated by the supervisors.
\begin{itemize}
	\item The time it takes for a ground handler to move from task A to task B could be underestimated
\end{itemize}
This case has some of the same consequences as the first one. If the ground handlers suddenly has less time to do what he needs to than he had anticipated, he will become stressed and the second task would be delayed amongst.
\begin{itemize}
	\item 
\end{itemize}