\section{Luggage}
Luggage is loaded on the plane using tugs, which transport containers with luggage. The Boeing 747 has seats for 416 passengers (http://alturl.com/w7bfj) and can carry roughly 6,500 kg of luggage. %Nedenstående kommer ude af sammenhæng - eller det er MEGET svært at vide hvad der menes
9,568 kg if plane would be entirely booked and every passenger had a 23 kg checked luggage, and in this estimate, the hand luggage is not taken into account. To transport such an amount of luggage, tied planning and careful transport is necessary in order to bring the luggage on the airplane in a timely fashion. %Ovenover - "tied planning", hvad menes?


Novia and SAS Ground Handling are two ground handling companies that have the responsibility of loading luggage(http://alturl.com/y9jcc). If a passenger's luggage is, by mistake, sent with a wrong plane, the passenger can contact the airline company, and then they will talk with the ground handling company, that handled the luggage. In Aalborg Lufthavn, luggage is equipped with a RFID chip that allows the airport to track the luggage, so as to make it easier to locate lost luggage.


Freight is a big part of aerial transportation and is an industry that existed as long as passenger transportation. although a lot of people have the notion that most freight is transported in airplanes for themselves but actually more than 60\% of all freight is transported taken on in the passenger flights in the unused space by passenger luggage.

"A growing proportion of world trade is now taken by air, and an increasing proportion of that is performed by the integrated, or express, carriers. These carriers offer an integrated door-todoor service with a guaranteed pickup and delivery time, relieving the customer of any direct involvement with customs and providing a tracking capability that allows the customer and the government inspection services to keep track of the shipment throughout the process. They mostly take shipments of less than 100 kg, but each of the four major players offers truly global coverage through intercontinental gateways, sub-hubs and fleets of delivery vans. Fedex, for example, delivers 3.2 million packages per day through 50,000 drop-off locations in more than 220 countries, using 671 aircraft, 41,000 vans and 138,000 employees."


"It is quite common for integrators to use space on combination carriers and vice-versa. There are also airlines that specialise in heavy lift, using small fleets of unique aircraft like the AN 124 or the Mil 10 helicopter."


"The trends also tend to reduce the ratio of value to weight, but the aircraft loads are still generally more limited by volumetric capacity than by weight limits."


"At airports in South East England, including Heathrow, each worker moves 67 tonnes per year compared with 22 tonnes in the Midlands airports [65]."


"Time in flight and in transit is most important, a saving of one hour perhaps being worth \$1000 in airport fees."


How to normally load:
"The ULDs are taken from the terminal on flat roller-bed dollys and loaded onto the aircraft main deck or belly holds using high-loader (Hi-Lo) vehicles fitted with driven roller beds.
Once inside the aircraft, they are transferred to roller beds on the floor of the aircraft, having been loaded in the correct order to achieve the necessary balance. Bulk cargo is loaded manually into the belly holds, having been brought in carts to the apron by tug and transferred to the aircraft door by self-powered conveyors (see Figure 11–4).

The specialized cargo handling equipment (e.g. Hi-Lo, see Figure 11–3) is very expensive and needs trained drivers. A careful balance needs to be struck between saving by sharing the equipment with the passenger apron or with other handling agents, and the possibility of letting the level of service slip below acceptable levels.
There needs to be enough artificial light on the apron to read documents, labels and placards, for the safety of the ground personnel while working in the midst of moving ground vehicles and aircraft servicing activity. However, it should not be so bright as to make it difficult for the flight crew to maneuver the aircraft.

There is a lot of movement of cargo between the passenger and cargo aprons at most airports.
It is therefore good practice for the distance to be kept as short as possible, within the framework of the airport’s master plan. An airside road connection of sufficient capacity should be provided, with the capability support 10 tonnes per axle, 12 metres wide, a maximum of 4\% gradient and a minimum turn radius of 20 m."

%TERMINAL DESIGN AND OPERATING CONSIDERATIONS side 223

"Integrators usually construct and operate their own terminal. Their traffic usually consists of packages less than 30 kg and courier mail. Activity is very peaky, and the dwell time is less than with conventional freight. Target service standards also help to determine the staffing and facility requirements. Typical standards may be:
 - consignments available for collection, examination or transhipment three hours after arrival
 - cleared consignments available within 15 minutes of consignee arriving at import collection point
 - customers to wait not more than 30 minutes after arrival for collection at truck dock
 - cargo reception to be complete within 30 minutes of arrival at truck dock."


"Most importantly, it is necessary to know whether the freight is to arrive and depart in bulk or on pallets or in small or large ULDs, and if the ULDs can be loaded direct from trucks that will be given access to the apron."


"All the above factors contribute to determining the level of mechanisation to be provided for
handling the freight. The choice is essentially between:
- manual: manpower plus fork lift trucks
- semi-mechanised: roller beds or conveyors
- fully mechanised: Elevating Transfer Vehicles (ETV), Automatic Storage and Retrieval Systems, Transfer vehicles.
A high labour content may mean that costs rise over time, but it is more possible to react flexibly to demand peaks. The fully mechanised approach only really works with high volumes of containerised freight and in a setting that can guarantee good maintenance skills. Even so, the whole terminal can come to a halt if an ETV breaks down. ETVs work up to seven metres high with hydraulic or electric chain drive. They are only applicable if there is a need to store a large number of ULDs over three decks. British Airways, at its multi-level World Cargo Centre at Heathrow, prefers lifts and lowered roller conveyors so that there are more options in case of breakdown. Transfer Vehicles are powered roller conveyors mounted on a carriage with bogies which hare electrically driven along rails in response to a command system."


"A semi-mechanised terminal may have belt conveyor systems and powered flat roller conveyors where the rollers are chain-driven from the previous one. They will also have reorienting and transfer dock beds: some have wheels that right angles rise up between the rollers, or powered ball decks, or heliroll rotation tables where the different quadrants are powered with a joystick."


"There are five main functions to be performed in the terminal:
- conversion between modes of transport
- sorting, including breaking down loads from originators and consolidating for destinations
- storage, and facilitating government inspection
- movement of goods from landside to airside and vice-versa, or from aircraft to aircraft
- documentation: submission, completion, transmission.
A good terminal will have systems that will allow efficient movement, effective storage, easy sortation, accurate and timely inventory control, tight security and effective use of manpower. Getting everything right can reduce the mis-handling rate from 1 : 20 to 1 : 26,000."


"Most freight comes from the shipper by road and leaves the airport by air, but it is not always so. Airlines use trucks with flight numbers instead of aircraft on relatively short haul sectors, and rail is also beginning to be used instead of road as a way of meeting sustainability targets. Also, there is an increasing trend to integrated air/road/rail/sea ports as at Dubai, Sharjah, Seattle. The freight on the trucks may be loose, on pallets, or already containerised. If it is from a ‘known’ shipper, the truck will have been sealed before leaving the shipper’s premises.
Unless there is a full container for one specific destination, the freight is taken out of the ULD or off the pallets and sorted by aircraft flight destination. All the freight for each destination is gathered together and stuffed into the container(s) appropriate for the aircraft being used. The same process applies to freight arriving by air for transfer to another aircraft, except that it comes into the terminal from airside rather than landside. The breakdown and stuffing is an entirely manual process, regardless of the degree of mechanisation in the terminal. It is good ergonomic practice to have height-adjustable platforms at the work stations. The platforms may also indicate the weight and sometimes the stability of the ULD. This information is vital for the correct loading of the aircraft."


"Storage is needed mostly for incoming freight which is awaiting clearance or collection following inbound clearance, but also for inbound freight before breakdown, outbound freight
awaiting consolidation or stuffing or awaiting departure, and transhipments. Collection can be a matter of an hour or two, but may, in some countries that have no strict policy for charging for it, extend to weeks as companies use the terminal as free storage. Typical times in the developed world are 20 hours for export, 40 hours for import and 32 hours for transhipment.
Traditionally, goods take six days from shipper to receiver. Approximately 90 per cent of this time is spent on the ground. Transport accounts for 12 per cent of this time. The rest can be
considered as delay, largely due to waiting for documentation to be completed due to lack of resources or information, or inaccurate delivery instructions, or problems with customs
clearance. ULDs vary in height from 1.7 m for lower deck, through 2.4 m for the main deck, to 3.0 m for fully contoured, these dimensions determining the rack spacing required for storing
them. Consignment storage is usually in one metre cubes of racking fed by fork lift or by automatic storage/retrieval systems as employed in supermarket warehouses. They may extend
to 20 units high by 30 long by 14 deep, to give 1000 cubic metres. Both inbound and outbound freight may be subject to inspection by government agencies for contraband, drugs, illegal
immigration, weapons, etc. Storage is much less necessary in integrators’ terminals, where the freight arrives only just in time for onward shipment and where almost all of it will have been pre-cleared though the EDI. Empty ULDs are often stored outside."


"Goods are usually moved from trucks into the terminal by trains of carts carrying bulk freight, pallets or containers pulled by tugs, preferably electrically powered. They are then manually off-loaded onto conveyers or taken by fork lift to be sorted by destination. Either the sort process deposits the goods directly at the stuffing platforms or they are again taken by conveyor or fork lift to the platform. The conveyor system for moving the goods inside the terminal can run to many kilometres in length. It is usually mounted high up to allow free movement of fork lifts and ULDs at ground level. Packages up to a maximum of 30 kg are put into trays on the conveyers. The filled containers are then moved to the airside dock designated for the aircraft by a system of roller beds, or omni-directional ball-top beds if there are changes in direction involved. In the reverse direction, a similar process of movement occurs, except that the trucks back up to doors in the terminal so that they can be loaded directly under a canopy rather than having to be fed by tugs or fork lifts."
