\section{Problem Statement}

Ground handling companies often hire low-paid workers, who work in an environment where they are exposed to congestion, stress, noise, jet-blast, extremes of weather and sometimes low visibility conditions. Stress is a very big part of the work in an airport, especially for the ground handlers, since airlines do not make money while the aircraft is not in the air; hence the ground handlers are very pressed on time.  In many places it is also the workers who are responsible for delays and in case of a delay can be deducted in salary.

When a worker is stressed he is more likely to make mistakes, which could lead to serious accidents. These accidents can first and foremost become dangerous for the workers because they can be hurt as a result of an accident. A survey made by ACI[citation needed] in 2004 showed that out of 15,119,020 aircraft movements 3,233 had accidents, concluding that 0.214\% of all turnovers had accidents.

Accidents do not only lead to dangerous situations for the workers, but can also become very expensive for the companies; first of all because of the cost of the repair, but also because the aeroplane will then have to spend more time on the ground.
\subsection{Problem Formulation}
\begin{center}
\textit{Most of the delays and errors that happen to aeroplanes are caused by the ground handlers, who service the planes. Is it possible to reduce stressfactors and optimize performance for ground handlers, by making an information system, that can dynamically manage ground handlers' tasks throughout the day? || Human errors and accidents during a turn-round is a result of stressed and unmotivated employees, causing delays, damage to equipment, loss of airtime and other unwanted annoyances for airlines and ground handling providers. Is it possible to formulate a software solution to resolve these issues by dynamically adapting to non-scheduled situations?}
\end{center}
