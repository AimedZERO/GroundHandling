\secion{State of the Art}
In the interview with Kim Bermann at Aalborg Airport, Kim mentioned which kind of different software Aalborg Airport uses.

Right now Aalborg Airport uses a program called TimePlan.
The program is a basic planning tool, where the administration has to either manually setup an entire week, and then repeat it, or they have the opportunity to copy a previous week.
If any changes occur during a week, the changes have to be done manually. Any change during a workday is changed on a piece of paper, which is hanging near the apron.
%Img: TimePlan.png

Every department have their own system, where each person gets a job description when the ground handler meets. The system tells the worker what the worker specific need to do throughout the day.

The way it work is the whole week is worked into TimePlan, then a timesheet where each airplane, and who is going to work on it is worked into a Microsoft Excel page, and then printed.

%Img: excelark.png

Every worker need to check in every day so the program can keep track of who is working the specific date, and who is not. TimePlan also comes with a application where workers can exchange who is work.

%Kilde, interview, og http://www.timeplan-software.com/

\subsecion{Peakhours}
\subsecion{Reaction to non-scheduled situations}


