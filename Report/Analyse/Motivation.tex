\section{Motivation}

In order to ensure the ground-handling employees work at their optimal performance, their individual motivation must be accounted for. Studies on this subject argues that; \"intrinsic motivation (based in interest) and autonomous extrinsic motivation (based in importance) are both related to performance, satisfaction, trust, and well-being in the workplace\" (Self-determination theory and work motivation, MARYLE`NE GAGNE ́AND EDWARD L. DECI).

image(Motivation.png)

It is important to maintain autonomous intrinsic motivation that an employee level of specific competence matches the requirements for that specific task, the task must not be too difficult or too simple. Failing to meet this will result in amotivation toward certain task because not really what one is doing or feeling too competent. Although tasks will incur in a groundhandling environment, an extrinsic motivation can promote autonomous behavior to get a raise or so the boss wont become upset.

image(Hackman.png)

Hackman and Oldham (1980) introduced a model of job characteristic, and suggested the most effective means of motivating individuals is through the optimal design of jobs.
The authors proposed that the means for increasing internal work motivation is to design jobs so they will:
(1) provide variety, involve completion of a whole, and have a positive impact on the lives of others;
(2) afford considerable freedom and discretion to the employee (what action theorists refer to as decision latitude); 
(3) provide meaningful performance feedback.

Constructive feedback is one way to influence autonomous motivation, but it also suggests that the interpersonal style of supervisors and managers is important.

Together, the studies suggest that autonomous motivation, consisting of a mix of intrinsic motivation and internalized extrinsic motivation, is superior in situations that include both complex tasks that are interesting and less complex tasks that require discipline. When a job involves only mundane tasks, however, there appears to be no performance advantage to autonomous motivation. Still, even in those situations, autonomous motivation will be associated with greater job satisfaction and well-being, as was found by Ilardi et al. (1993) in a study of employees with monotonous jobs in a shoe factory and by Shirom and colleagues (1999) in a study of blue-collar workers with mundane jobs. This implies that, overall, autonomous motivation is preferable in organizations because even with dull, boring jobs there is an advantage to autonomous motivation in terms of job satisfaction and well-being, which are likely to yield better attendance and lower turnover (Breaugh, 1985; Karasek & Theorell, 1990; Matteson & Ivancevich, 1987; Sherman, 1989).


