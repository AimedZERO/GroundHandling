\chapter{Stakeholders}
\section{Organisations}
\subsection{Ground Handling Companies}
Ground handling companies are one of the primary stakeholders to consider, since they are the main target group for our project.

Ground handling companies provide an array of services for the airport and airlines. Among these are aircraft handling services, such as cabin cleaning, loading and unloading, luggage sorting and ULD control. Besides handling the aircraft, many also handle passenger services, such as check-in, lost and found and VIP services. Some ground handling companies even assist with weather briefing or flight operations.

For the ground handling companies to handle such a vast amount of services, they need a large number of employees carrying out a lot of different tasks. Managing all of these employees is therefore not an easy task, and can require multiple supervisors, making sure that each of the ground handlers is working efficiently and that they are not idle for prolonged periods of time. The job of managing this crew of ground handlers is usually done manually, where the supervisors distribute the workload among the ground handlers to the best of their ability, and assigning tasks to the ground handlers individually.
This method is far from ideal, as mistakes can easily happen when assigning tasks among large numbers of ground handlers, due to the supervisors being pressed on time, as to not delay the work of the ground handlers. These mistakes can potentially be catastrophic as they can lead problems ranging from delays, due to poor planning, to crucial errors in handling aircraft, due to assigning a ground handler to a task, which he does not have the required skills to perform.

To further clarify which specific problems arise when delegating tasks among the ground handlers, we will conduct an interview with a/some airport(s). These interviews and the results thereof will be described later.