\chapter{Stakeholders}
In this chapter the different stakeholders for this project will be described. The focus will be on three categories of stakeholders, organisations, management and operational personnel. The main focus will be on the operational personnel, as these are judged to have the largest interest and use of this project.
\section{Organisations}
This section concerns the organisations which are relevant to this project. The section's main focus will be which jobs and protocols there are in the organisations and how they differ.
\subsection{Ground Handling Companies}
Ground handling companies are one of the primary stakeholders to consider, since they are the main target group for our project.

Ground handling companies provide an array of services for airports and airlines. Among these are aircraft handling services, such as cabin cleaning, loading and unloading, luggage sorting and ULD (Unit Load Device) control. Besides handling the aircraft, many companies also handle passenger services, such as check-in, lost and found and VIP services. Some ground handling companies even assist with weather briefing and flight operations.

For the ground handling companies to handle such a vast amount of services, they need a large number of employees carrying out a lot of different tasks. Managing all of these employees is not an easy job, and can require multiple supervisors, making sure that each of the ground handlers is working efficiently and that they are not idle for prolonged periods of time. The job of managing this crew of ground handlers is usually done manually, where supervisors distribute the workload among the ground handlers and assign tasks to the ground handlers individually, to the best of their ability.
This method is far from ideal, as mistakes can easily happen when assigning tasks among large numbers of ground handlers, due to the supervisors being pressed on time, as to not delay the work of the ground handlers. These mistakes can potentially be catastrophic as they can lead to problems ranging from delays, due to poor planning, to crucial errors in handling aircraft, to assigning a ground handler to a task in which he does not have the required skills to perform.

To further clarify which specific problems arise when delegating tasks among the ground handlers, there will be conducted an interview with an airport. This interview and the results thereof will be described later.

%

Aalborg airport is in charge of both the ground handlign and checkin for all companies execpt SAS, who is in chage of checkin of their own airplanes. It depends on the size of the airport who is in charge of the ground handling and checking, Billund and Aalborg airport is in chage by them self, bigger airports like Copenhagen or Amsterdam, are only in chage of the runways and securety. For the rest they have other ground handling companies.
Until an airport exceeds 2.000.000 passengers per year, they can keep all ground handling for them selfs. If they exceeds 2.000.000 passengers per year can outside companies make offers, and if they are better, the airport cannot refuse.
Aalborg airport has around  1.400.000 passengers per year

The services airplanes can have and the contracts that each company sign, can vary alot. In the IATA Airport Handling Manual all the codes for each of the numbers in a contract can be found, in the IATA Airport Handling Manual each code/number is described. Though compaies uses old contracts, so the codes can vary, the IATA Airport Handling Manual is published every 5 year.

%

In conclusion the many ground handlers performs a high variety of tasks in which they are administered by their supervisors. If the supervisors are able to retain their overview of the ground handling crews, the amount of mistakes and crucial errors are tied to together by the fact, that tasks and jobs are delegated manually to the individual workers.