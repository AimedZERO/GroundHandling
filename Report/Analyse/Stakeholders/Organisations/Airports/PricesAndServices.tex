\subsection{Airports}

In the following section, airports, mainly Aalborg Airport, will be described. THe section will go into the prices and services of Aalborg Airport, describing prices on things like fuel. In the section it will also be described how the emergency protocols of big airorts are, and how these are supposed to be handles. Furthermore the section look at the types and amount of airplanes travveling through Aalborg Airport, and at last it will be described how the teams at Aalborg airport is managed, and how this could be bettered.

\subsubsection{Prices and Services}
When designing products aimed at airports, it is interesting to find out what services and what prices the airports usually deal with. In this section we will look at different prices and services in airports.


Fueling is an important part of ground handling, since refueling it is a task that must be performed every time a planes arrives. Aalborg Airport has an agreement with the Shell corporation, to get their fuel supply from them. The air planes are fueled with a fuel type called 100LL \cite{iaopa_fuelprices}, and is a very common aircraft fuel, it is priced at DKK 19.85 pr. litre, which means that if you would have to fill up a Boeing 737-800, which can contain 26,020 litres\cite{737_specs}, it would cost DKK 516,497 plus the start up fee.

As described above it is clear that there is a lot of money going around in an airport, even when only one air plane is taken into account. The air planes need to get filled up before every lift off, since a Boeing 737-800 uses 3,200 litres of fuel pr. hour when it is in the sky. This means that if you where to fly from Aalborg to Copenhagen it would cost, just in terms of fuel, DKK 47,640.


According to Aerohandlers' pricelist (%source http://www.aerohandlers.com/download/aerohandlers_pricelist.pdf)
, airlines pay upwards of USD 3000 for a single airline to have ground handling performed upon. This indicates that the cost of performing ground handling, for the ground handling companies, is also very high. If you were able to optimize ground handling procedures, then costs for the ground handling companies could be reduced, leading to lower prices for the airlines and ultimately lower prices for the passengers. Therefore, optimizing the work flow on ground handling procedures is a very relevant aspect of designing solutions for aiports.