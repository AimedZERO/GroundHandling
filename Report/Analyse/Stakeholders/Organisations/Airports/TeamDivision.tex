\subsubsection{Team Division}
In order to make a system that is helpful for the airport, it is important to know the people who supposedly are going to use it and how they work. In this section the team division will be described in order to get a clear view on the properties of the teams at Aalborg Airport.

All information used in this section have been collected from the interview made at Aalborg Airport.



At Aalborg Airport a schedule will be made on a weekly basis, which contains information on what tasks the workers are assigned to. The workers are divided into teams that will be responsible for the different departures and take offs. How the teams are composed may vary from day to day as the workers are able to perform various tasks. A worker who is taught how to pushback may also be able to do the departure checks and so on. After five to six years, a worker should be able to do almost all of the different tasks that are present at the airport. This is valued by the airport as they believe the workers is more satisfied when their tasks are mixed up once in a while, in order to keep them challenged. It is different from an airport as Copenhagen Airport where the workers are designated only one or two tasks.  


In short, the workers at the airport are not always in the same teams or assigned to the same tasks, as they can handle a wide variety of tasks, and because the airport wishes for the workers to be challenged.