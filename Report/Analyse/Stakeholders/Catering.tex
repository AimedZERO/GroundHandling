\section{Catering} 
When passengers are on long flights, there will be served air meals for them which consists of different kinds of meat, vegetables and drinks. The pilots is served with the same dishes, although this variates in between the different\cite{cate_pilotfood1}\cite{cate_pilotfood2}. SAS offers breakfast when flying domestic flights, but only from 6.00 AM to 9.00 AM\cite{cate_SASIndri}. When flying in between Scandinavian and European countries there will always be served coffee or tea, but in order to get breakfast served the passenger has to be a member of premium service\cite{cate_SASscanda}.

All these air meals and drinks are prepared and made by the catering services that are a part of the ground handling scheme. The catering companies are chosen by the different airlines individually. SAS for example, have used Gate Gourmet\cite{cate_SASGourmet} as their catering service for a number of years. The catering services are very difficult in the sense of how complex the logistic aspect is as the President of KLM Catering said "Flight catering is 70 
percent logistics and 30 per cent cooking”\cite{cate_BookSection}. Therefore the catering services want their operations to go as smoothly as possible. The airline companies are also very interested in getting capable catering services as of them use the food as a marketing technique\cite{cate_BookSection}. 