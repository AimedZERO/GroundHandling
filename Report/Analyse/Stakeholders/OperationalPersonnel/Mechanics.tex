\subsection{Air plane Mechanic}
To be able to make a thorough analysis of the personal involved in the handling and maintenance of the flights, it's necessary to take a look at the relevant workers. Mechanics are the workers handling maintenance, and in this chapter their job and the potential for optimization of this, will be evaluated.

The airplane mechanic has a very important job in the airport, their job is to repair the airplanes. It's therefore important that these working crews can access the airplane that needs repairing quick and easy, and know where they needs to go, so the airplane can get flying again as quickly as possible. It is also important that they do their job as good as possible, as every part needs to be maintained correctly.

In Scandinavia and the Baltic countries, the leading airplane mechanic company regarding technical airplane maintenance is SAS Tech\cite{sas_tech_mechanic}.

In Aalborg Airport the method of which the planes are currently gone through for mistakes, is actually not a technician going through the planes known error locations, but it is often a baggage handler with a specific course, so he knows exactly what to look at, at specific places on the airplane.

The way the scheduling of the airplane reparation happens is that the airtraffic companies themselves repair the airplanes. Aalborg Airport is not big enough for the airtraffic companies to have repairs on their planes made there, and therefore it must be set up so the planes will land in another larger airport at a specific time for airplane to get maintained. The maintenance of the airplanes often happen in the night time.

If a airplane has trouble and need to be repaired while at Aalborg Airport, a technician will fly in to Aalborg Airport to repair the airplane. This happens a couple of times a year, but as mentioned earlier, schedueled maintanence of the airport does not happen at Aalborg Airport.

If it would be possible for airplane mechanics to know exactly where to go by viewing it on a smartphone, PDA or other portable device, the mechanics would be able to be more productive, as they would know instantly where they where needed for their next task. 

Boeing has released an application to help the mechanics get important things like airplane manuals, and serial numbers for specific parts. If they would be able to do this with every airplane, or every airplane at Aalborg Airport, it would help the mechanics tremendously as they would be able to identify the parts they needed much quicker, and therefore the maintenance of an airplane would be quicker. By using an application you would also be able to find earlier maintenance records, meaning that if another mechanic may have made a mistake, you would be able to identify it much quicker\cite{cnet_boeing_app}.

A downside to making an application like Boeing, could be the obvious problem of Grease. If the Mechanics are supposed to use a tablet, PDA or smartphones, wouldn't they need to use a lot of time to clean their hands in between using the device, and would this really make the repairing of the aircraft faster?
%Kan vi undgå at skrive "obvious"? Måske noget ala "The same problem as GRease have suffered".
%"and would this really make the repairing of the aircraft faster?" Det ved jeg ikke vil det det? 

Other suggestions for application that could help mechanics do their job faster an better include:
\begin{itemize}
\item An application that gives suggestions how to fix things by describing the problem (Ex. Weird Noise in Cabin)
\item An application that would help them calculate different things, like what to setting to set the torque screwdriver to. 
\end{itemize}

After this has been checked it can be concluded that the only area that can be easily implemented in a software solution, would be the manuals and serial numbers part.
