\section{Operational Personnel}
\subsection{Freight}
In this section the freight handled at airport will be analysed as the transportation of goods and passengers plays a major role in a everyday life of an airport. 

Luggage is loaded on the plane using tugs, which transport containers with luggage. The Boeing 747 has seats for 416 passengers\cite{freight_boing} and can carry roughly 6,500 kg of luggagem or 9,568 kg if the plane would be entirely booked and every passenger had a 23 kg checked luggage, as in the estimate the hand luggage is not being taken into account. To transport such a huge amount of luggage, tight planning and careful transport is necessary in order to bring the luggage onto the airplane in a timely fashion.

Novia and SAS Ground Handling are two ground handling companies that have the responsibility of loading luggage\cite{mistet_bagage}. If a passenger's luggage is, by mistake, sent with a wrong airplane, the passenger can contact the airline company, and then they will talk with the ground handling company that handled the luggage. In Aalborg Airport, luggage is equipped with a RFID chip that allows the Airport to track all luggage, and use that information in an advanced sorting machinery that has managed to almost eliminate loss of luggage.


Luggage isn't the only thing transported. Aside from passengers, cargo is a big part of aerial transportation and is an industry that has existed as long as passenger transportation. An increasingly growing part of the world trade is beginning to be transported by air and although a lot of people have the notion that most cargo is transported in separate airplanes, actually more than 60\% of all cargo is transported in passenger flights in the unused space besides passenger luggage.


Besides the passenger flights an increasing proportion of cargo is transported by integrated (where the airlines or other companies have their own equipment and airplanes) or express (where it is handled as additional luggage) carriers by a so-called door-to-door service, where the company transport the goods all the way. Since the companies that transport the cargo is in charge of all the transportation, both in air and on ground, the tracking of cargo is a lot easier and the direct involvement of the costumer is kept to a minimum. Their services mostly take shipments less than 100 kg. This service help the larger companies transport cargo a lot easier. For instance FedEx delivers 3.2 million packages per day in more than 220 countries through 50,000 drop-off locations, using 671 aircraft, 41,000 vans and 138,000 employees (2005).

Many integrators, companies using integrated carriers, construct and operate their own terminal where their goods arrive and is checked, packed, documented, transported to the apron and so on by their own system. 
Their traffic is normally very peaky and the dwell time is normally shorter. Their goods normally consists of packages smaller than 30 kg and courier mail. At these terminals the standards are normally:
\begin{itemize}
\item Consignments available for collection, examination or transhipment(ready for collection) three hours after arrival
\item Cleared consignments available within 15 minutes of consignee arriving at import collection point
\item Customers to not wait more than 30 minutes after arrival for collection at truck dock
\item Cargo reception to be complete within 30 minutes of arrival at truck dock
\end{itemize}

When cargo arrives at the airport it normally arrives at a terminal, it is normally transported via electrical tugs from the trucks into the terminal in carts carrying bulk cargo, pallets or containers. The cargo is now taken through a sorting process that deposits the goods directly at the stuffing platforms or they are again taken by conveyor (packages up to a maximum of 30 kg are put into trays on the conveyers) or fork lift to the platform.
Unless the container for a destination is full, the cargo is rearranged at these platforms by destination in new containers called ULDs, which stands for Unit Load Device and is normally a pallet or container, specific for the aircraft type it needs to be transported on.


This also applies to cargo arrived from air from another airplane where the cargo is in transit in the current airport. The only difference being that this cargo arrives from the air side, not the land side.
This process of rearranging is entirely manual, no matter how mechanized the terminal is (will be described shortly) and is preferably done on height-adjustable platforms that can indicate the weight and sometimes the stability of the ULD. This information is very important when you load the aircraft to ensure a stable aircraft in balance.


There are five different tasks performed in the terminal:
\begin{itemize}
\item Conversion between modes of transport
\item Sorting, including breaking down loads from originators and consolidating for destinations
\item Storage, and facilitating government inspection
\item Movement of goods from landside to airside and vice-versa, or from aircraft to aircraft
\item Documentation: submission, completion, transmission
\end{itemize}

Getting these five tasks just right and performed smoothly and effectively can reduce the mishandling rate from 1 : 20 to 1 : 26,000.

Normally the terminals use the storage area to store cargo, which is awaiting clearance, but it is also used for cargo before it is rearranged or outbound cargo awaiting consolidation, stuffing or simply waiting for its departure time and transhipments. This pickup can be a matter of an hour or two, but can in some countries be several weeks, if they have no restrictions since it is essentially free storage for the companies. This, in developed worlds, is normally not the case, where the time is normally 20 hours for export, 40 hours for import and 32 hours for transhipment. %er dette afsnit på nogen måde relevant?? samme med de følgende 3

In total, an order takes about 6 days from sender to receiver, where the cargo normally spends 90\% of its time on the ground where 12\% is transport time and the rest is storage, where the cargo is waiting on documentation due to lack of resources or information, inaccurate delivery instructions or problems with customs clearance. This stands very much in contrast to the inside of the integrators' terminal where cargo normally arrives just before it is time to be shipped by plane and has already been cleared and sorted.%mange procenter uden kilder

All cargo can at all times be forced to be inspected by government agencies for contraband, drugs, illegal immigration, weapons and so on.

In different countries they have different standards for labour and the level of automation in the terminal is therefore different in each country. Generally there are three different levels of mechanization.
\begin{itemize}
\item Manual: Manpower plus fork lift trucks
\item Semi-mechanised: Roller beds or conveyors
\item Fully mechanised: Elevating Transfer Vehicles (ETV), Automatic Storage and Retrieval Systems, Transfer vehicles
\end{itemize}

The benefits of a manual and mostly labour controlled system is that it is flexible in peak-hours and can easily adjust but the  disadvantages is that is is more expensive over time.
On the other hand, a fully mechanised system functions best when a lot of cargo flows through the terminal and all of this is containerised and the machines can be serviced very fast. Of cause the whole system can still break down if an ETV (Elevating Transfer Vehicle; the vehicles that organise multilevel storageup to seven meters) breaks down.  Therefore for instance British Airways, also use lifts and lowered roller conveyors at it's multi-level World Cargo Centre at Heathrow, in case of a breakdown.

Normally cargo is transported to the flights at the so-called aprons which is the area where the flight is serviced by the ground handlers, usually via trucks, though some airports also use rail.

This cargo and luggage is normally transported in ULDs via roller-bed dollys (Flat carts acting as wheels for the ULDs) to the aircraft and then lifted into the aircraft either from the side or the front via high-loader vehicles (A truck specialised to raise and move the ULDs inside the aircraft).
The ULD can now be organised inside the aircraft on roller beds (a small track-like system consisting of small cylindrical "wheels" where the ULD can be pushed). The cargo needs to be loaded in the right order to achieve balance. The bulk cargo (cargo that is not containerized or on pallets), which have been transported to the flight in carts, can now be loaded into the flight via self powered conveyer belts. Therefore is it very important for the airport to know if the cargo will arrive in bulks, on pallets or on ULDs and if it needs transportation from the terminal to the apron, or the company will transport it on trucks, granting access to the aprons.
%Et billed af de ULD's vil være godt?

%Nedenstående: Hvad er "this equipment"?
This equipment is very expensive, especially the high-loaders, which needs specialised drivers; as a result, the companies often share this equipment.

There is a lot of movement between the cargo and passenger aprons and they should therefore be placed close to each other.%relevant?

Research:
\begin{itemize}
\item How much cargo is normally transported?
\item How expensive is it to have you airplane in the airport in fees? ("Time in flight and in transit is most important, a saving of one hour perhaps being worth \$1000 in airport fees. (2005)")
\end{itemize}


Knowledge (2005):
\begin{itemize}
\item The trends also tend to reduce the ratio of value to weight, but the aircraft loads are still generally more limited by volumetric capacity than by weight limits.
\end{itemize}

In conclusion; the amount of freight, consisting of cargo, mail and passengers, are present everyday in multiple tons. The personal and ULD's are working precisely  in order to make sure the freight is transported to the right locations and flights. The cargo can not wait for too long at the storing areas as there is a certain time limit at  most airfields. A fully mechanised system will be able to service containers and machines in a very fast manner, as a lot of these flows through the terminals.
