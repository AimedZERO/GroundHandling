\chapter{Airports}
\section{Prices and Services}
When designing products aimed at airports, it is interesting to find out what services and what prices the airports usually deal with. In this section we will look at different prices and services in airports.


Fueling is an important part of ground handling, since refueling it is a task that must be performed every time a planes arrives. Aalborg Airport has an agreement with the Shell corporation, to get their fuel supply from them. The air planes are fueled with a fuel type called 100LL \cite{iaopa_fuelprices}, and is a very common aircraft fuel, it is priced at DKK 19.85 Pr. litre, which means that if you would have to fill up a Boeing 737-800, which can contain 26,020 litres\cite{737_specs}, it would cost DKK 516,497 plus the start up fee.

As described above it is clear that there is a lot of money going around in an airport, even when only one air plane into account. The air planes need to get filled up before every lift off, since a Boeing 737-800 uses 3,200 litres of fuel pr. hour when it is in the sky. This means that if you where to fly from Aalborg to Copenhagen it would cost, just in terms of fuel, DKK 47,640.

[INSERT INFO ABOUT GROUND HANDLING COMPANIES]
According to Aerohandlers' pricelist (source http://www.aerohandlers.com/download/aerohandlers_pricelist.pdf), airlines pay upwards of USD 3000 for a single airline to have ground handling performed upon. This indicates that the cost of performing ground handling, for the ground handling companies, is also very high. If you were able to optimize ground handling procedures, then costs for the ground handling companies could be reduced, leading to lower prices for the airlines and ultimately lower prices for the passengers. Therefore, optimizing the work flow on ground handling procedures is a very relevant aspect of designing solutions for aiports.

%\section{Luggage}
Luggage is loaded on the plane using tugs, which transport containers with luggage. The Boeing 747 has seats for 416 passengers(http://alturl.com/w7bfj) and can carry roughly 6,500 kg of luggage. %Nedenstående kommer ude af sammenhæng - eller det er MEGET svært at vide hvad der menes
9,568 kg if plane would be entirely booked and every passenger had a 23 kg checked luggage, and in this estimate, the hand luggage is not taken into account. To transport such an amount of luggage, tied planning and careful transport is necessary in order to bring the luggage on the airplane in a timely fashion. %Ovenover - "tied planning", hvad menes?


Novia and SAS Ground Handling are two ground handling companies that have the responsibility of loading luggage(http://alturl.com/y9jcc). If a passenger's luggage is, by mistake, sent with a wrong plane, the passenger can contact the airline company, and then they will talk with the ground handling company, that handled the luggage. In Aalborg Lufthavn, luggage is equipped with a RFID chip that allows the airport to track the luggage, so as to make it easier to locate lost luggage.
%Kilde til l4 