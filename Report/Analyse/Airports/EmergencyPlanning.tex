\section{Emergency Protocols}

Occasionally unexpected emergencies occur, which the airport needs to respond to. A standard service manual for handling potential emergencies exists. When designing systems for airports it is relevant to know how they handle emergency landings: Which runways to shut down and prepare for the emergency, how to handle incoming and outgoing traffic and other airport services.
The manual suggest following plan for an aircraft accident on the airport:



In general a lot of different organizations is involved in these emergencies, each with their own responsibilities. The airport traffic services includes following:

\begin{quotation}
Chapter 4
RESPONSIBILITY AND ROLE OF EACH AGENCY FOR EACH TYPE OF EMERGENCY

4.1 AIRCRAFT ACCIDENT ON THE AIRPORT
4.1.1 General
The airport emergency plan shall be implemented immediately upon an aircraft accident occurring on the airport. For this
type of emergency, responding agencies are expected to take action as described in 4.1.2 to 4.1.10 below.
4.1.2 Action by air traffic services
4.1.2.1 Initiate emergency response by using the crash alarm communication system (See Figure 8-1).
4.1.2.2 Notify the rescue and fire fighting service and provide information on the location of the accident, grid map
reference and all other essential details, including time of the accident and type of aircraft. Subsequent notification may
expand this information by providing details on the number of occupants, fuel on board, aircraft operator, and any
dangerous goods on board, including quantity and location, if known.
4.1.2.3 Close the affected runway and minimize vehicle traffic on that runway to prevent disturbance of accident
investigation evidence (See 4.1.5 2) f)).
4.1.2.4 If required, initiate communications to the police and security services, airport authority, and medical
services in accordance with the procedure in the airport emergency plan. Provide the contacts with grid map reference,
rendezvous point and/or staging area and airport entrance to be used.
4.1.2.5 Issue the following Notice to Airmen (NOTAM) immediately:
“Airport rescue and fire fighting service protection unavailable until (time) or until further notice. All equipment committed
to aircraft accident.”
4.1.2.6 Verify by written checklist that the actions above were completed, indicating notification time(s) and name
of person completing action.



4.3 FULL EMERGENCY
4.3.1 General
The agencies involved in the airport emergency plan shall be alerted to “full emergency” status when it is known that an
aircraft approaching the airport is, or is suspected to be, in such trouble that there is a possibility of an accident.
4.3.2 Action by air traffic services
4.3.2.1 Notify the airport rescue and fire fighting service to stand by at the predetermined ready positions
applicable to the planned runway and provide as many of the following details as possible:
a) type of aircraft;
b) fuel on board;
c) number of occupants, including special occupants — handicapped, immobilized, blind, deaf;
d) nature of trouble;
e) planned runway;
f) estimated time of landing;
g) aircraft operator, if appropriate; and
h) any dangerous goods on board, including quantity and location, if known.
4.3.2.2 Initiate notification of the mutual aid fire department(s) and other appropriate organizations in accordance
with the procedure prescribed in the airport emergency plan, providing, if necessary, the rendezvous point and airport
entrance to be used.



4.4 LOCAL STANDBY
4.4.1 General
The agencies involved in the airport emergency plan shall be alerted to “local standby” status when an aircraft
approaching the airport is known or is suspected to have developed some defect but the trouble is not such as would
normally involve any serious difficulty in effecting a safe landing.
4.4.2 Action by air traffic services
Notify the airport rescue and fire fighting service to stand by as requested by the pilot, or stand by as local airport
agreements require at the predetermined ready positions applicable to the runway to be used. Provide as many of the
following details as possible:
a) type of aircraft;
b) fuel on board;
c) number of occupants, including special occupants — handicapped, immobilized, blind, deaf;
d) nature of trouble;
e) planned runway;
f) estimated time of landing;
g) aircraft operator, if appropriate; and
h) any dangerous goods on board, including quantity and location, if known.
\end{quotation}


In conclusion, a runaway can be assigned to "full emergency" or "local standby" statuses, and when an accident occurs, the affected area is closed and traffic through the area is minimized. Furthermore, a signal of NOTAM is issued to notify that airport rescue and fire fighting services are all currently occupied.



-source
Airport Services Manual, Part 7 by International Civil Aviation Organization (ICAO)
Second Edition - 1991

